% Chapter Template

\chapter{Introduction} % Main chapter title

\label{Chapter1} % Change X to a consecutive number; for referencing this chapter elsewhere, use \ref{ChapterX}

%----------------------------------------------------------------------------------------
%	SECTION 1
%----------------------------------------------------------------------------------------

\section{Motivation}
In the last twenty years, the number of mobile devices in use has tremendously increased. In the first quater of 2018 more than 380 Million smartphones have been sold worldwide \cite{Gartner}. However, in the past few years, not only smartphones have been sold, but also a new market of mobile gadgets and connected devices, summed up as Internet of Things, has evolved. In 2017, more than 20 Billion devices were connected to the internet. Forecasts predict 30 Billion devices in 2020 and already more than 70 Billion in 2025. \cite{Statista}

This increase in mobile computing has also increased the demand of accurate real-time positioning systems which led to an active research mainly in indoor positioning system technologies, as there are established solutions for outdoor positioning. 

%-----------------------------------
%	SUBSECTION 1
%-----------------------------------
\subsection{Indoor difficulties vs Outdoor}

For outdoor applications, primarily the Global Positioning System (GPS) is in use. For indoor application in the other hand, GPS has limitations that make it almost useless. Due to the environmental conditions indoors, with heavy walls armoured with steel and other distractions, additional signal loss is encountered which makes it hard to detect and decode GPS signals. \cite{GPSforIndoor} In addition, higher buildings in the neighborhood can reflect transmitted signals, which leads to false position estimations. As GPS is mainy applied as 2D positioning system, it will not provide 3D indoor information such as the current floor level.
For this purposes we are forced to use alternative technologies. There are many different approaches to do indoor positioning, which made it an attractive and active research field. 

%-----------------------------------
%	SUBSECTION 2
%-----------------------------------

\subsection{Important Applications}
Morbi rutrum odio eget arcu adipiscing sodales. Aenean et purus a est pulvinar pellentesque. Cras in elit neque, quis varius elit. Phasellus fringilla, nibh eu tempus venenatis, dolor elit posuere quam, quis adipiscing urna leo nec orci. Sed nec nulla auctor odio aliquet consequat. Ut nec nulla in ante ullamcorper aliquam at sed dolor. Phasellus fermentum magna in augue gravida cursus. Cras sed pretium lorem. Pellentesque eget ornare odio. Proin accumsan, massa viverra cursus pharetra, ipsum nisi lobortis velit, a malesuada dolor lorem eu neque.

%----------------------------------------------------------------------------------------
%	SECTION 2
%----------------------------------------------------------------------------------------

\section{Ideas}

Sed ullamcorper quam eu nisl interdum at interdum enim egestas. Aliquam placerat justo sed lectus lobortis ut porta nisl porttitor. Vestibulum mi dolor, lacinia molestie gravida at, tempus vitae ligula. Donec eget quam sapien, in viverra eros. Donec pellentesque justo a massa fringilla non vestibulum metus vestibulum. Vestibulum in orci quis felis tempor lacinia. Vivamus ornare ultrices facilisis. Ut hendrerit volutpat vulputate. Morbi condimentum venenatis augue, id porta ipsum vulputate in. Curabitur luctus tempus justo. Vestibulum risus lectus, adipiscing nec condimentum quis, condimentum nec nisl. Aliquam dictum sagittis velit sed iaculis. Morbi tristique augue sit amet nulla pulvinar id facilisis ligula mollis. Nam elit libero, tincidunt ut aliquam at, molestie in quam. Aenean rhoncus vehicula hendrerit.


%-----------------------------------
%	SUBSECTION 1
%-----------------------------------
\subsection{Ranging Positioning System with different Inputs}

Nunc posuere quam at lectus tristique eu ultrices augue venenatis. Vestibulum ante ipsum primis in faucibus orci luctus et ultrices posuere cubilia Curae; Aliquam erat volutpat. Vivamus sodales tortor eget quam adipiscing in vulputate ante ullamcorper. Sed eros ante, lacinia et sollicitudin et, aliquam sit amet augue. In hac habitasse platea dictumst.


%----------------------------------------------------------------------------------------
%	SECTION 3
%----------------------------------------------------------------------------------------

\section{Contributions}

Lorem ipsum dolor sit amet, consectetur adipiscing elit. Aliquam ultricies lacinia euismod. Nam tempus risus in dolor rhoncus in interdum enim tincidunt. Donec vel nunc neque. In condimentum ullamcorper quam non consequat. Fusce sagittis tempor feugiat. Fusce magna erat, molestie eu convallis ut, tempus sed arcu. Quisque molestie, ante a tincidunt ullamcorper, sapien enim dignissim lacus, in semper nibh erat lobortis purus. Integer dapibus ligula ac risus convallis pellentesque.

%-----------------------------------
%	SUBSECTION 1
%-----------------------------------
\subsection{Implementation of the System}

Nunc posuere quam at lectus tristique eu ultrices augue venenatis. Vestibulum ante ipsum primis in faucibus orci luctus et ultrices posuere cubilia Curae; Aliquam erat volutpat. Vivamus sodales tortor eget quam adipiscing in vulputate ante ullamcorper. Sed eros ante, lacinia et sollicitudin et, aliquam sit amet augue. In hac habitasse platea dictumst.



%-----------------------------------
%	SUBSECTION 2
%-----------------------------------
\subsection{Comparison of three different implementations}

Nunc posuere quam at lectus tristique eu ultrices augue venenatis. Vestibulum ante ipsum primis in faucibus orci luctus et ultrices posuere cubilia Curae; Aliquam erat volutpat. Vivamus sodales tortor eget quam adipiscing in vulputate ante ullamcorper. Sed eros ante, lacinia et sollicitudin et, aliquam sit amet augue. In hac habitasse platea dictumst.


