% Chapter Template

\chapter{Implementation and experiment setup} % Main chapter title

\label{Chapter4} % Change X to a consecutive number; for referencing this chapter elsewhere, use \ref{ChapterX}
In this section we explain in detail, what hardware we used in our implementation and with which technology the UWB communication and ranging was done. Finally we briefly explain the core of our work, the mathematical theory of the particle filter.

%----------------------------------------------------------------------------------------
%	SECTION 1
%----------------------------------------------------------------------------------------

\section{Hardware}
We decided to use Raspberry Pis for all mobile devices such as the TAG and the ANs. As the Raspberry Pis were not equipped with UWB technology, we extended them with a SEQUITUR Pi board from UNISET company running InGPS Lite firmware. In the following two subsections we present you these two hardware components.

%-----------------------------------
%	SUBSECTION 1
%-----------------------------------

\subsection{Raspberry Pi}
A Raspberry Pi is a single-board computer not much bigger than a credit card. Raspberry Pis are mainly designed for educational purposes as an alternative to expensive notebooks or desk computers. Hence the focus lies also on easy-to-use and plug-and-play experiences. Raspberry Pis are useful for versatile types of projects, as they provide common state of the art hardware - like HDMI, USB and wireless LAN - direct on boad and as they are extendable with selected components.

We used Raspberry Pi Model B \cite{Raspberry}, these were the most relevant specifications for our work:
\begin{itemize}
\item Quad Core 1.2GHz Broadcom BCM2837 64bit CPU
\item 1GB RAM
\item 100 Base Ethernet
\item 40-pin extended general purpose input output (GPIO)
\item Micro SD port for loading your operating system and storing data
\end{itemize}

%-----------------------------------
%	SUBSECTION 2
%-----------------------------------
\subsection{SEQUITUR Pi board with InGPS Lite}
On the 40-pin extended GPIO, we connected the SEQUITUR Pi board from UNISET Company. UNISET is a company located in Italy that focuses on research, development and manufacturing of innovative sensors in two major application areas \cite{Uniset}:

\begin{itemize}
\item Access control security systems, enhancing the reliability of intrusion detection
\item Indoor and outdoor tracking. Sequitur is a precise real time locating system (RTLS) for tracking any object in 2D or 3D with centimeter accuracy.
\end{itemize}

This hardware seemed perfect for our ambitions, as it provides a state-of-the-art UWB communication and ranging. Moreover the SEQUITUR Pi board of the TAG has IMU sensors like 3D-accelerometer and 3D-magnetometer on board. Together with the hardware, UNISET delivers a firmware running on Raspberry Pis operating system (OS) to establish a connection via user datagram protocol (UDP). This firmware allows to communicate with the sensors in order to retrieve IMU sensor data, but also to get direct access to the range between two nodes. It is explained in detail throughout the next section. 

%----------------------------------------------------------------------------------------
%	SECTION 2
%----------------------------------------------------------------------------------------

\section{UWB Communication and Ranging}
The radio module of SEQUITUR Pi board is used on the one hand to transmit data - in order to obviate the need for additional communication hardware - and on the other hand to evaluate the ToF. As UNISET is a comercial company, they do not provide full information of the underlying techniques. Nonetheless in the two following subsections, the known parts are mentioned.

%-----------------------------------
%	SUBSECTION 1
%-----------------------------------
\subsection{Transmission}
SEQUITUR InGPS Lite enables single-hop wireless communication with the UWB interface between neighboring nodes of the same network. 
The radio module supports six different user-selectable frequency bands between 3.5 GHz and 6.5 GHz. There are six different operation modes to change the spectral occupation, listed in the table below:\\
\\
\begin{tabular}{c c c}
Channel Number  & Central Frequency [MHz] & Bandwidth [MHz]\\
1 & 3494.4 & 500\\
2 & 3993.6 & 500\\
3 & 4492.8 & 500\\
4 & 3993.6 & 1300\\
5 & 6489.6 & 500\\
7 & 6489.6 & 1100\\
\end{tabular}
\\
\\
The data rate can be changed to three preset values of 110 kilobit per second (kbps), 850 kbps and 6.8 megabit per second (Mbps). All nodes have to operate in the same radiomode to communicate correctly. A lower data rate allows lager operating distances between the nodes. 
The default pulse repetition frequency (PRF) is assumed to 64 MHz for all the channels.
The underlying modulation techniques are not indicated in the specifications. \cite{Usermanual} \cite{Beginnersguide}

%-----------------------------------
%	SUBSECTION 2
%-----------------------------------
\subsection{Ranging with TWR}
UNISET company offers two different packages for positioning. InGPS Lite, which is the standard software and InGPS Pro, which is the advanced package. For our implementation InGPS Lite was sufficient, as the main difference of the two packages are the number of TAGs and anchors supported. With InGPS Lite only one TAG and a maximum of 10 anchors are supported as with InGPS Pro numerous TAGs and anchors are possible. InGPS Lite opperates only in TWR mode other than InGPS Pro, where a second mode with TDOA range estimation is available.
The range estimation of two nodes is triggered by the application programming interface (API) command $CLIENT\_GET\_RANGE\ (50)$. In our application we sent the command to the anchor in order to minimize the communication of the TAG. The flow of actions related to this API is a even more simplified version of the message exchange indicated in Figure \ref{fig:two_way_ranging}. In our case, the request message performed by the client starts the TWR conversation via UWB between AN and TAG. The AN sends only one ranging request to the TAG, which immediately responds. By observing the difference between the time instants related to the transmission of the request packet and the reception of the response packet, the AN will directly determine the RTT and thus the range. Finally an answer message with the range is reported from the AN to the client and no messages are reported from the TAG to the client.

%----------------------------------------------------------------------------------------
%	SECTION 3
%----------------------------------------------------------------------------------------

\section{Particle Filter}
We used the a particle filter approach to solve the localization problem. This method, also known as Monte Carlo Localization (MCL), is often used for indoor positioning. It combines various noisy measurements to estimate the system state and minimize errors. To introduce the particle filter, we explain in a first paragraph which inputs we fused into the particle filter. The following paragraphs define the different phases in our mathematical model whereas we discuss the different variations of our system in the last subsection.

%-----------------------------------
%	SUBSECTION 1
%-----------------------------------
\subsection{Inputs}
As stated above, various measurements are taken into account in our particle filter. The most important inputs are the range estimations between the TAG and ANs, the motion vector measured by the IMU of the TAG and the restrictions given by the floormap. We will refer to $Zd_{t}$ as the range observation vector at time t, which is described as $Zd_{t} = [d^{j}_{t}], j = 1...N,$ where N is the number of ANs. Every distance measurement $d^{j}_{t}$ itself is consisting of various errors, statistically it can be described as: 
$$d^{j}_{t} = \hat{d}^{j}_{t} + d^{j}_{be, t} + \epsilon_{d^{j}, t},$$
where $\hat{d}^{j}_{t}$ is the actual distance to node j, $d^{j}_{be, t}$ is an environmental bias due to local conditions (obstacles) and $\epsilon_{d^{j}, t},$ is a measured random error.

The motion vector $Mv_{t} = [\theta_{t}, \ell_{t}]$ is modeled by the heading direction $\theta$ and the movement length $\ell$. Both of $\theta$ and $\ell$ are given by the IMU, where again several noises occur such that the heading direction is statistically described as:
$$\theta_{t} = \hat{\theta}_{t} + \theta_{bs,t} + \theta_{be,t} + \epsilon_{\theta, t},$$
with $\hat{\theta}_{t}$ as the actual heading orientation,  $\theta_{bs,t}$ as a sensor bias introduced by uncalibrated sensor readings, $\theta_{be,t}$ as an environmental angular bias due to magnetic field disturbances and $\epsilon_{\theta, t}$ as a measured random error. 

Whereas the heading direction is directly calculated from IMU sensors, for the stride length we took the previous system state into account. This was necessary, because measured errors propagate over time, so it is almost impossible to use relative quantities (e.g. acceleration) to calculate absolute quantities (e.g. distance) over a longer period of time.
We approximated the observed movement length as follows:
$$\ell_{t} = \delta_{t-1} + \frac{a_{t-1} + a_{t}}{2} * \Delta t^{2}$$
where $\delta_{t-1}$ is the distance the estimated position moved in the system state, $a_{t}, a_{t-1}$ the measured acceleration at time t, respectively time t-1 and $\Delta t$ the time passed between t and t-1. Due to sensor data update frequency we assume the change in acceleration during the time period as constant. The movement length can be defined as: 
$$\ell_{t} = \hat{\ell}_{t-1} + (\hat{a}_{t} + a_{bs,t} +\epsilon_{a, t}) * \Delta t^{2} + \epsilon_{\ell, t},$$
where $\hat{\ell}_{t-1}$ is the actual movement length of time period t-1, $\hat{a}_{t}$ is the actual middle acceleration during time period t, $a_{bs,t}$ is an other sensor bias due to uncalibrated sensor readings and $\epsilon_{a, t}$ as well as $\epsilon_{\ell, t}$ are measured random errors in acceleration and distance respectively. 

For the likelihood calculation in our work we nevertheless used the actual measurements $d^{j}_{t}$, $\theta_{t}$ and $\ell_{t}$ since the errors are handled in the likelihood model by fusing different data sources and anchor node distances. However, to compensate the bias and error for the particle spreading, we assumed the heading direction $\theta$ and the stride length $\ell$ as random normal variables whose values are obtained from $\mathcal{N}(\theta_{t}, \sigma_{\theta}^{2})$ and $\mathcal{N}(\ell_{t}, \sigma_{\ell}^{2})$.



%-----------------------------------
%	SUBSECTION 2
%-----------------------------------
\subsection{Prediction phase}
Each particle has a state vector that is defined as follows:
$$X_{t} = [x_{t}, y_{t}, x_{t-1}, y_{t-1}]$$
where $(x_{t}, y_{t})$ corresponds to the Cartesian coordinates of the particle at time t and $(x_{t}, y_{t})$ at time t-1 respectively. In the prediction phase each particle is updated depending of the current movement vector $Mv_{t} = [\theta_{t}, \ell_{t}]$. The coordinates of the particle are updated with the following pattern:
$$[x_{t}, y_{t}]  = [x_{t-1} + \ell_{t} * cos(\theta_{t}),\quad y_{t-1} + \ell_{t} * sin(\theta_{t})]$$
As mentioned in the last subsection, with $\ell_{t}$ and $\theta_{t}$ as random normal variables. In the remainder of this work we will also refer to the motion in Cartesian coordinates as $M_{x,t} = \ell_{t} * cos(\theta_{t})$ for the motion in x-direction and $M_{y,t} = \ell_{t} * sin(\theta_{t})$ for the motion in y-direction.
Floorplan restrictions are applied in this phase, whereas movements through walls are not permitted, they lead to another prediction iteration for that particle.

%-----------------------------------
%	SUBSECTION 3
%-----------------------------------
\subsection{Observation phase}
In the observation phase an associated weight $w^{i}_{t}$ is recalculated for every partical, since the weight does not anymore correspond to the current position. The weight is updated corresponding to the likelihood of the range observations conditioned on each particle $p(Zd_{t} | X^{i}_{t})$ at time t, respectively the likelihood of the motion observation conditioned on each particle $p(Mv_{t} | X^{i}_{t})$ at time t. Then, the probability is determined as:
$$ p(Zd_{t} | X^{i}_{t}) = p(d_{t}^{j} | X^{i}_{t}) $$
and $$ p(Mv_{t} | X^{i}_{t}) = p(M_{x,t} | X^{i}_{t}) * p(M_{y,t} | X^{i}_{t})$$ %is this correct that p(x) and p(y) are multiplied?
In order to avoid confusion between different likelihoods used in this work, hereafter we refer to $p(d_{t} | X^{i}_{t})$ as the ranging likelihood, $p(M_{t} | X^{i}_{t})$ for the motion likelihood and $p(Z_{t} | X^{i}_{t})$ as the overall likelihood. $z_{t}$ is used when it does not matter if $d_{t}$ or $M_{t}$ is inserted.


likelihood:
$$p(\hat{d}_{j,t} | X^{i}_{t}) = \frac{1}{2*\pi \sigma_{Mx}} * exp(\frac{-[\sqrt{(x^{i}_{t}-x_{j})^{2}+(y^{i}_{t}-y_{j})^{2}} - \hat{d}_{j,t}]^{2}}{2*\sigma_{Mx}})$$
$$p(\hat{M}_{x,t} | X^{i}_{t}) = \frac{1}{2*\pi \sigma_{Mx}} * exp(\frac{-[(x^{i}_{t}-x^{i}_{t-1}) - \hat{M}_{x,t}]^{2}}{2*\sigma_{Mx}})$$

%-----------------------------------
%	SUBSECTION 4
%-----------------------------------
\subsection{Resampling phase}


%-----------------------------------
%	SUBSECTION 5
%-----------------------------------
\subsection{Variants}
Variants of the system (one without spreading particles with movementvector and one without likelihood)


%----------------------------------------------------------------------------------------
%	SECTION 3
%----------------------------------------------------------------------------------------

\section{Experiment Setup}


\subsection{Likelihood and weighting}
Here goes the text.


%-----------------------------------
%	SUBSECTION 3
%-----------------------------------
\subsection{Ranging with TWR}
Here goes the text.