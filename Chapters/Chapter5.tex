% Chapter Template

\chapter{Experiments evaluation} % Main chapter title

\label{Chapter5} % Change X to a consecutive number; for referencing this chapter elsewhere, use \ref{ChapterX}
In this section the results of our experiments are presented. In the first two sections we compare the estimated positions of the three tested algorithms. Firstly the results of experiments with wide spread anchors are presented, secondly the results with nearer anchor positions are shown. In a third section we justify the differences between these two scenarios. In the last section we conclude our work and propose next improvement steps for our algorithm. 

%----------------------------------------------------------------------------------------
%	SECTION 1
%----------------------------------------------------------------------------------------

\section{Indoor positioning results with low density of anchor nodes}
We used the follwing configurations for our experiments:
\begin{itemize}
\item Frequency band was set to parameter X: With blabla and so on MHz and middle of xx. 
\item Data rate was set to 2, which corresponds to 850 kbps
\item Transmission power txpower is set to 63 (highest value)
\end{itemize}

%-----------------------------------
%	SUBSECTION 1
%-----------------------------------

\subsection{Raspberry Pi}
A Raspberry Pi is a single-board computer not much bigger than a credit card. Raspberry Pis are mainly designed for educational purposes as an alternative to expensive notebooks or desk computers. Hence the focus lies also on easy-to-use and plug-and-play experiences. Raspberry Pis are useful for versatile types of projects, as they provide common state of the art hardware - like HDMI, USB and wireless LAN - direct on boad and as they are extendable with selected components.



%-----------------------------------
%	SUBSECTION 2
%-----------------------------------
\subsection{SEQUITUR Pi board with InGPS Lite}
On the 40-pin extended GPIO, we connected the SEQUITUR Pi board from UNISET Company. UNISET is a company located in Italy that focuses on research, development and manufacturing of innovative sensors in two major application areas \cite{Uniset}:

\begin{itemize}
\item Access control security systems, enhancing the reliability of intrusion detection
\item Indoor and outdoor tracking. Sequitur is a precise real time locating system (RTLS) for tracking any object in 2D or 3D with centimeter accuracy.
\end{itemize}
Here goes the text.

%----------------------------------------------------------------------------------------
%	SECTION 2
%----------------------------------------------------------------------------------------

\section{Indoor positioning results with high density of anchor nodes}
Here goes the text.

%-----------------------------------
%	SUBSECTION 1
%-----------------------------------
\subsection{Transmission}
Here goes the text.
%-----------------------------------
%	SUBSECTION 2
%-----------------------------------
\subsection{Ranging with TWR}
Here goes the text.

%----------------------------------------------------------------------------------------
%	SECTION 3
%----------------------------------------------------------------------------------------

\section{Comment}
Here goes the text.
\begin{itemize}
\item Spread the particles and validate new positions with floormap constraints
\item Evaluate UWB ranges, IMU measures and zone indication to assign likelihood
\item Calculate weight function and systematically resample (and reposition) particles with low weights
\item Sum up weighted positions
\end{itemize}

%-----------------------------------
%	SUBSECTION 1
%-----------------------------------
\subsection{Inputs}
Here goes the text.



%-----------------------------------
%	SUBSECTION 2
%-----------------------------------
\subsection{Likelihood and weighting}
Here goes the text.


%-----------------------------------
%	SUBSECTION 3
%-----------------------------------
\subsection{Ranging with TWR}
Here goes the text.



%----------------------------------------------------------------------------------------
%	SECTION 4
%----------------------------------------------------------------------------------------

\section{Conclusion and further work}
Here goes the text.
\begin{itemize}
\item Spread the particles and validate new positions with floormap constraints
\item Evaluate UWB ranges, IMU measures and zone indication to assign likelihood
\item Calculate weight function and systematically resample (and reposition) particles with low weights
\item Sum up weighted positions
\end{itemize}
